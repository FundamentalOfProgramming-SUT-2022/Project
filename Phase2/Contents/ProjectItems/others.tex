\subsubsection*{{\titr دستورات فاز اول}}
\addcontentsline{toc}{subsubsection}{{\fehrestContent دستورات فاز اول}}


این فاز هیچ کدام از دستورات فاز اول را منسوخ نمی‌کند؛ به این معنی که تمام دستورهایی که در فاز قبل وجود داشتند مثل insertstr یا arman یا... در این فاز هم به همان شکل وجود دارند. حتی دستوری مثل find که در این فاز به نوع جدیدی مطرح شده، باید به صورت فاز اول هم قابل‌استفاده باشد.
تنها تغییری که در این دستورات به وجود می‌آید این است که یک کاراکتر ":" به ابتدای این دستورات اضافه می‌شود.

\ybox{مثال}{:createfile file\_name.txt}

خروجی این دستورات (در صورت وجود) باید به صورت یک فایل بدون نام در ویرایشگر باز شوند. پس مثلا اگر دستور

\ybox{مثال}{:tree}

وارد شد، لازم است فایل فعلی در صورت وجود بسته شود، و یک فایل بدون نام با محتوای درخت مشابه فاز اول باز شود. واضح است که امکان جستجو، تغییر و ذخیره‌سازی مثل هر فایل دیگری برای این فایل هم تعریف شده است. با این وجود در صورتی که این فایل بدون ذخیره‌سازی بسته شد (یا فایل جدیدی باز شد) لازم است هیچ اثری از این فایل باقی نماند.
