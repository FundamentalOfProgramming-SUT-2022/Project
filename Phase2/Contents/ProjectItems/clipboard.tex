
\subsubsection*{{\titr Clipboard}}
\addcontentsline{toc}{subsubsection}{{\fehrestContent Clipboard}}


پس از انتخاب قسمتی از متن در حالت
visual
باید این امکان برای کاربر وجود داشته باشد که متن انتخاب شده را حذف، قیچی و یا روگرفت بکند. کاربر باید بتواند در حالتی که متن انتخاب شده، به وسیله هر کدام از دستورات زیر عملیات مورد نظر خودش را انجام بدهد.

\begin{latin}
    \begin{itemize}
        \item{Cut/Delete}: d
        \item{Copy}: y
    \end{itemize}
\end{latin}

بدین منظور باید یک فیچر clipboard برای ادیتور خود پیاده‌سازی کنید تا قسمت‌هایی که cut/copy می‌شوند در آن قرار بگیرند. پس از هر کدام از کلیدهای مربوط به روگرفت و یا قیچی باید از حالت visual به حالت normal منتقل شوید.
توجه کنید که دستورات قیچی و روگرفت فاز اول هم در همین دسته محسوب می‌شوند و فرقی نمی‌کند که از کلیدها و یا دستورات فاز اول برای اضافه کردن متن به clipboard استفاده شده باشد.

دستور paste که در حالت normal تعریف می‌شود
به کاربر این امکان را می‌دهد که در این حالت بتواند به وسیله کلید p مقدار ذخیره‌شده در clipboard برنامه (که حاصل اجرای یک دستور cut یا copy بوده است) را در محل نشان‌گر (cursor) قرار دهد.

\newpage
