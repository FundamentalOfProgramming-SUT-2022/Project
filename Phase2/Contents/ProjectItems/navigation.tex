\subsubsection*{{\titr Navigation}}
\addcontentsline{toc}{subsubsection}{{\fehrestContent Navigation}}

برنامه‌ی شما باید قابلیت حرکت دادن نشان‌گر (cursor) با استفاده از کیبورد بر روی متون را داشته باشد. مواردی که باید پیاده سازی به آنها توجه کنید:
\begin{itemize}
    \item کاربر باید بتواند نشان‌گر (cursor) را بر روی بخشی از متن که در حال نمایش است به چپ و راست و بالا و پایین حرکت دهد. (کلیدهای پیشنهادی: بالا(k) پایین(j) چپ(h) و راست(l))
    \item با رسیدن به ابتدای هر خط، با فشردن کلید چپ و همچنین با رسیدن به انتها خط و فشردن کلید راست، نباید اتفاقی بیفتد.
    \item انتقال به خط بعد/قبل (بالا و پایین رفتن) باید به مکان نسبی مشابه مکان نسبی نشان‌گر (cursor) در خط فعلی باشد. (برای مثال اگر در خط فعلی در کاراکتر بیستم قرار داریم، با فشردن کلید j باید در خط بعد و    کاراکتر بیستم قرار داشته باشیم.) اگر تعداد کاراکتر خط بعدی کمتر بود، به آخرین کاراکتر آن انتقال یابیم.
    \item در ۴ خط مانده به پایان صفحه فعلی، با فشردن کلید حرکت به پایین، باید یک خط شیفت دهید. یعنی یک خط جدید از فایل به صفحه اضافه شده و خط اول حذف شود. همچنین وقتی نشان‌گر (cursor) در محل ۴ خط مانده به ابتدای صفحه فعلی است، در صورت فشردن کلید بالا، باید یک خط از بالای صفحه اضافه شده و یک خط از پایین حذف شود.
    \item لازم است توجه کنید با کلید پایین رفتن باید به خط بعدی با توجه کاراکتر \textbackslash n بروید و شکل ظاهری خطوط ملاک نیست.
\end{itemize}
