\section*{{\titr مقدمه}}
\addcontentsline{toc}{section}{{\fehrestContent مقدمه}}

\subsection*{{\titr اهداف پروژه}}

\addcontentsline{toc}{subsection}{{\fehrestContent اهداف قابل توجه}}

\begin{itemize}

\item
هدف این پروژه طراحی یک ابزار ویرایش فایل مشابه vim است. احتمالا در کارگاه کامپیوتر و یا جاهای دیگر با این ابزار کار کرده‌اید. در غیر این صورت می‌توانید از طریق \href{https://www.openvim.com}{\textcolor{blue}{\underline{این لینک}}} نحوه کار با این ابزار را ببینید.

\item
در این فاز از برای ابزاری که در فاز قبل طراحی کردید رابط کاربری می‌سازید و برخی ویژگی‌ها را بهبود می‌دهید.

\item
در این پروژه نحوه پیاده‌سازی اجزای مختلف از اهمیت بسیاری برخوردار است و تنها خروجی نهایی مهم نیست. از این رو برای تمیزی کد خود ارزش قائل شوید. 

\item
آشنایی با سیستم مدیریت نسخه \lr{Git} و کار بر روی پروژه بر بستر یک مخزن \lr{Github}، یکی از اهداف مهم پروژه است. در این مورد توصیه می‌شود تغییرات خود را در دوره‌های کوتاه مدت \lr{commit} کنید.

\end{itemize}

\subsection*{{\titr کلیات پروژه}}
\addcontentsline{toc}{subsection}{{\fehrestContent کلیات پروژه}}

در این فاز، کد فاز اول خود را کامل می‌کنید.

در ادامهٔ مستند، موجودیت‌ها، نمای کلی رابط کاربری سیستم، نقش‌ها و دستورات لازم شرح داده‌شده است.

\begin{enumerate}[label={نکته \arabic*:}]
\item
در هر جایی از پروژه می‌توانید هرگونه خلاقیتی را به‌کار ببرید. با این حال توجه کنید که خواسته‌های واضح پروژه بایستی انجام شوند و سیستم ورودی گرفتن و خروجی دادن شما باید مطابق جزییات گفته شده در این مستند باشد.


\item در این فاز برخی از دستورات را با استفاده کلیدهای شورتکات پیاده‌سازی می‌کنید. کلیدهای ذکرشده در ادامه‌ی مستند پیشنهادی هستند و می‌توانید به صلاح‌دید خود آن‌ها را تغییر دهید. توجه کنید که این کار نباید منجر به محدود شدن کاربری برنامه‌ی شما شود.
% \newpage
% \item
% در مستند بعضی از دستور‌هایی که مشاهده می‌کنید فرمتی به شکل زیر دارند:

% \begin{mybox}[colback=yellow]{دستور}
	
	
% 	\begin{latin}
		
% 	compare file1 file2
		
% 	\end{latin}
	
% \end{mybox}
% همچنین توجه کنید که اگر دستور نامعتبری وارد شد که در قسمت مربوطه، برای آن خطای به خصوصی در نظر گرفته نشده بود، پیام زیر را چاپ کنید:

% \begin{mybox}[colback=yellow]{پیغام به کاربر}
	
	
% 	\begin{latin}
		
% 	invalid command
		
% 	\end{latin}
	
% \end{mybox}


\end{enumerate}
